%*********************************************************************
% gdutthesis: 广东工业大学论文模板
% 2024/01/19 v0.1f
%
% 重要提示:
%   1. 请确保使用 UTF-8 编码保存
%   2. 请使用 XeLaTeX 或 LuaLaTeX 编译
%   3. 请仔细阅读用户文档和 Wiki
%   4. 修改、使用、发布本文档请务必遵循 LaTeX Project Public License
%   5. 不需要的注释可以尽情删除
%   6. 请使用最新版本的模板并在打印前检查输入信息与学校要求是否一致
%*********************************************************************
\documentclass[
  % type=doctor
  type=master
  % type=promaster
]{gdutthesis}

% 宏包在这里加载
\usepackage{siunitx}[=v2]
\usepackage{zhlipsum,lipsum}

\gdutsetup{
  style = {
    % cover           = {true},
    cover           = {false},
    open            = {true},
    % open            = {false},
    free-float      = {true},
    % free-float      = {false},
    % font            = {garamond},
    % font            = {libertinus},
    % font            = {lm},
    % font            = {palatino},
    font            = {times},
    % font            = {times*},
    % math-font       = {garamond},
    % math-font       = {libertinus},
    % math-font       = {lm},
    % math-font       = {palatino},
    math-font       = {times},
    % math-font       = {times*},
    cjk-font        = {fandol},
    % cjk-font        = {founder},
    % cjk-font        = {mac},
    % cjk-font        = {sourcehan},
    % cjk-font        = {noto},
    % cjk-font        = {windows},
    % cjk-font        = {none},
    bib-backend     = {bibtex},
    % bib-backend     = {biblatex},
    bib-resource    = {ref/gdutthesis-template.bib, ref/reference.bib},
    bib-style       = {numerical},
    % bib-style       = {author-year},
    % fullwidth-stop  = {mapping},
    % fullwidth-stop  = {catcode},
    fullwidth-stop  = {false},
    hyperlink       = {color},
    % hyperlink       = {border},
    % hyperlink       = {none},
    hyperlink-color = {default},
    % hyperlink-color = {autumn},
    % hyperlink-color = {business},
    % hyperlink-color = {classic},
    % hyperlink-color = {elegant},
    % hyperlink-color = {fantasy},
    % hyperlink-color = {material},
    % hyperlink-color = {science},
    % hyperlink-color = {summer},
    % hyperlink-color = {graylevel},
    % hyperlink-color = {prl},
  },
  theorem = {
    header-font = bf,
    % header-font = sf,
    body-font   = rm,
    % body-font   = kai,
    indent      = cn,
    % indent      = en,
    interval    = dash,
    % interval    = dot,
    braces      = paren,
    % braces      = bracket,
    punct       = colon,
    % punct       = quad,
    qed         = \qedsymbol,
    % qed         = \QED,
  },
  info = {
    title             = {标题},
    title*            = {Title},
    date              = {2020/5/25},
    author            = {张三},
    author*           = {ZHANG San},
    supervisor        = {李四, 教授},
    supervisor*       = {LI Si},
    supervisor-two    = {none},
    supervisor-two*   = {none},
    supervisor-three  = {none},
    supervisor-three* = {none},
    department        = {自动化学院},
    department*       = {Automation},
    major             = {控制科学与工程},
    student-id        = {2112101234},
    chairman          = {赵六\quad 教授},
    degree            = {工学硕士},
    degree*           = {Master of Engineering Science},
    keywords          = {关键词1, 关键词2, 关键词3, 关键词4},
    keywords*         = {keywords 1, keywords 2, keywords 3, keywords 4},
    secret-level      = {none},
    delay-time        = {none},
  }
}


\begin{document}

\gdutstatement% 空白声明页
% \gdutstatement[日期][作者签名][导师签名]% 签名为图片
% \gdutstatement*[扫描件]% 扫描件为 A4 大小的 pdf 文件

\input{data/abstract}

\input{data/notation.tex}

\printnoidxglossaries

\gduttableofcontents

\mainmatter

\chapter{绪论}{Introduction}

\section{本课题研究背景及研究意义}{Background and significance of research}

\zhlipsum[1]

\section{国内外相关研究现状}{Analysis of the research status at home and abroad}

\subsection{测试}{test}

测试\gdutcite{chendengyuan2000guoshi},测试\gdutcite*{woerdelun2012jingji}。

测试如\autoref{fig:example} 所示,具体参考\autoref{sub-fig-1} 和\autoref{sub-fig-2}。

测试参考\autoref{eq:example}。

测试参考\autoref{tab:example}。

测试\gls{slm},\gls{slm},\gls{glm},\gls{glm}。

\subsubsection{测试}
测试 test。
\paragraph{测试 test。}
测试 test。
\subparagraph{测试 test。}
测试 test。

% 大部分情况使用 align/gather 环境,小部分间距异常情况可尝试使用 equation 环境
\begin{align}
  E &= mc^2 \label{eq:example} \\
  mc^2 &= E
\end{align}

\begin{gather}
  E = mc^2 \\
  mc^2 = E
\end{gather}

\begin{figure}
  \subfloat[贴有模板的金属喷嘴示意图]{\label{sub-fig-1}
    \includegraphics[width=0.4\textwidth]{example-image.pdf}
  }
  \qquad
  \subfloat[由点到线扫描加工原理图]{\label{sub-fig-2}
    \includegraphics[width=0.4\textwidth]{example-image.pdf}
  }
  \bicaption{模板射流电解加工微沟槽原理图}{Principle of masked jet electrochemical machining of micro grooves}
  \label{fig:example}
\end{figure}

\begin{table}
  \bicaption{DMC5400A 运动控制卡主要技术指标}{DMC5400A main specifications}
  \label{tab:example}
  \begin{tabular}{cc}
    \toprule
    控制卡技术指标              & 具体参数                      \\
    \midrule
    控制电机的脉冲信号频率范围  & $\SI{1}{Hz}\sim\SI{2}{MHz}$   \\
    控制电机的脉冲信号频率精度  & \SI{0.0625}{Hz}               \\
    脉冲信号输出最大电流        & \SI{20}{mA}                   \\
    脉冲信号长度                & 28 位有符号                   \\
    直线插补精度                & $\pm \SI{0.8}{pulse}$         \\
    圆弧插补精度                & $\pm \SI{1.5}{pulse}$         \\
    支持的插补坐标系个数        & 2                             \\
    \bottomrule
  \end{tabular}
\end{table}

\input{data/conclusion.tex}

\gdutbackmatter

\nocite{*}% 列出 bib 文件中的全部参考文献
\printbibliography

\chapter{攻读学位期间取得与学位论文相关的成果}{Publication and patents during study}

\gdutbacksection{发表和投稿与学位论文相关学术论文}

% 成果内容可以用 bibtex 生成后,从 main.bbl 里复制到 results
\begin{results}
  \item 
  \textbf{ZHANG S}, LI S, WANG W, et~al.
  \newblock
  Jet electrochemical machining of micro dimples with conductive mask\allowbreak[J/OL].
  \newblock
  Journal of Materials Processing Technology, 2018, 257: 101-111.
  \newblock
  DOI: \doi{10.1016/j.jmatprotec.2018.02.035}.
  \newblock
  (SCI Impact Factor: 6.162, JCR: Q2, 中科院分区: 材料科学 2 区, WOS: 000431161400010)
  \item 
  \textbf{ZHANG S}, LI S, WANG W, et~al.
  \newblock
  Electrochemical direct-writing machining of micro-channel array\allowbreak[J/OL].
  \newblock
  Journal of Materials Processing Technology, 2019, 265: 138-149.
  \newblock
  DOI: \doi{10.1016/j.jmatprotec.2018.10.014}.
  \newblock
  (SCI Impact Factor: 6.162, JCR: Q2, 中科院分区: 材料科学 2 区, WOS: 000451935100014)
\end{results}

\gdutbacksection{申请和授权发明专利}

\begin{results}
  \item 
  李四, \textbf{张三}, 王五.
  \newblock
  一种微流道电解加工装置.
  \newblock
  发明专利申请号: 201810467763.5.
  \item 
  李四, \textbf{张三}, 王五.
  \newblock
  一种双脉冲变极性同步电解的系统与方法: 201910620061.0\allowbreak[P].
  \newblock
  2020-10-09.
\end{results}

\chapter{致谢}{Acknowledgements}
\zhlipsum[1]

\gdutappendix

\input{data/appendix.tex}

\end{document}