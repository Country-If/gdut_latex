\documentclass[
  type=doctor
]{../gdutthesis}

\usepackage{siunitx}[=v2]

\usepackage{glossaries}
\renewcommand{\loadglsentries}[1]{}

\gdutsetup{
  style = {
    cover           = {false},
    open            = {true},
    font            = {times*},
    math-font       = {none},
    cjk-font        = {windows},
    bib-backend     = {bibtex},
    bib-resource    = {gdutthesis-template.bib},
    bib-style       = {numerical},
    fullwidth-stop  = {false},
    hyperlink       = {none},
  },
  info = {
    title             = {模板射流电解加工微沟槽关键技术研究},
    title*            = {Investigation on masked jet electrochemical machining of micro grooves},
    date              = {2020/5/25},
    author            = {张三},
    author*           = {ZHANG San},
    supervisor        = {吕四, 教授},
    supervisor*       = {LYU Si},
    supervisor-two    = {none},
    supervisor-two*   = {none},
    supervisor-three  = {none},
    supervisor-three* = {none},
    department        = {机电工程学院},
    department*       = {Electromechanical Engineering},
    major             = {工程(机械工程)},
    student-id        = {},
    chairman          = {赵六\quad 教授},
    degree            = {工程硕士},
    degree*           = {Master of Engineering},
    keywords          = {电解加工, 微沟槽, 模板, 射流},
    keywords*         = {electrochemical machining, micro grooves, mask, jet},
    secret-level      = {none},
  }
}

\setmathfont{Cambria Math}

\begin{document}

\begin{abstract}
  微沟槽作为一种典型的表面微结构,在微传感器、微反应器、微流体器件以及微
  型燃料电池等产品核心零部件上发挥重要作用。如何实现复杂形状微沟槽的精密、高
  效、低成本制造成为研究的热点。目前,微沟槽的加工方法主要有微细机械切削加工、
  微细激光加工、微细电火花加工以及微细电解加工等加工方式。电解加工因具有无应
  力接触、无切削热、无工具电极损耗、可实现离子级别的材料去除等特点,在金属微
  结构加工方面优势明显。针对现有掩模电解加工中掩模制作工艺复杂、电解产物排除
  困难、沟槽长度方向尺寸一致性差等问题,本文提出一种模板射流电解加工微沟槽结
  构新方法。该方法将带有单个或单排微通孔结构的模板与金属喷嘴集成为一体,加工
  过程中模板与工件表面保持接触,电解液通过喷嘴高速喷入模板微通孔到达工件表面,
  通过控制喷嘴与工件之间的相对运动便可由点到线扫描加工出复杂曲线形或阵列微沟
  槽结构。本文的主要研究内容如下。

  (1)提出模板射流电解加工方法,搭建电解加工微沟槽实验平台,完成模板喷嘴
  与工装夹具设计制作,开发基于 LabVIEW 软件的运动控制系统和数据采集系统,以满
  足实际加工实验需求。

  (2)通过 ANSYS 电场仿真对模板射流电解加工微沟槽成形规律进行理论分析。
  分别完成了单点微坑和微沟槽成形过程模拟,通过计算电流密度变化过程分析了微沟
  槽加工首尾两端圆角形成过程,并将理论计算与实际加工微沟槽首尾两端形貌进行对
  比。研究了加工电压、模板尺寸与扫描速度参数对微沟槽仿真计算结果的影响,为进
  一步实验与工艺参数优化打下了基础。

  (3)进行模板射流电解加工微沟槽工艺实验,研究了关键参数如脉冲频率、脉冲
  占空比、喷嘴扫描速度以及扫描次数与微沟槽加工定域性、材料去除率以及电流效率
  之间的关系。最终通过参数优选,在加工电压30 V,脉冲频率1 000 Hz,占空比20\%,
  扫描速度20 μm/s的条件下成功加工出槽宽90 μm,槽深20 μm的弯曲蛇形、直线和交叉
  阵列形以及阵列微缝等高质量微沟槽结构,验证了模板射流电解加工方法的可行性。

  (4)为提高微沟槽模板射流电解加工定域性,提出采用金属模板代替绝缘模板进
  行电解加工。仿真结果表明,采用金属模板能够改变微沟槽加工区域电场分布,使模
  板孔边缘处电场削弱,从而减小侧向腐蚀。实验结果表明,相比绝缘模板,采用金属
  模板加工侧向腐蚀系数 $F_{\symup{E}}$ 从 1.47 直接增加至 6.11。采用金属模板能够显著减小射流电
  解加工过切量,提高加工定域性。
\end{abstract}

\begin{abstract*}
  As a typical micro scale structure, micro grooves with a certain size and shape are
  widely used in micro sensors, micro reactors, micro fluidic devices and micro fuel cells.
  How to realize the precision, high efficiency and low cost manufacturing of micro grooves
  with complex shapes has become the focus of research. At present, the machining methods
  of micro groove mainly include mechanical micromachining, laser beam micromachining,
  electrical discharge micromachining and electrochemical micromachining. Electrochemical
  machining has obvious advantages in metal microstructure processing due to its features of
  stress-free contact, no cutting heat, no electrode tool wear, and ion level material removal.
  Aiming at the problems such as complex mask making process, difficult to eliminate
  electrolytic products, and poor consistency of groove size in traditional electrochemical
  machining, this thesis proposed a novel approach of masked jet electrochemical machining
  for micro groove structure. A flexible insulated mask with micro through holes was covered
  on the head surface of a metallic nozzle. During machining, the mask on the modified nozzle
  was contacted with the workpiece, and the jetting electrolyte in the nozzle was divided into
  different machining regions by the micro-through-holes in the mask, then, the micro grooves
  could be generated by moving the workpiece with an effective voltage applied between the
  nozzle and workpiece. By controlling the relative motion between nozzle and workpiece,
  complex curved shape or array micro grooves can be processed from dot to line scanning.
  The main contents of this thesis are as follows.

  (1) A masked jet electrochemical machining method was proposed, and an experimental
  device for micro groove machining was built. The software development of motion control
  and data acquisition based on the LabVIEW platform was carried out to meet the
  requirements of the processing experiment.

  (2) In this thesis, the forming process of micro grooves in masked jet electrochemical
  machining is theoretically analyzed by ANSYS electric field simulation. The simulation of
  single-point micro dimple forming process and the simulation of point-line micro groove
  forming process were completed respectively. The process of forming transition arcs at both
  starting and ending points of micro groove was analyzed by calculating the current density
  change, and the arc profiles of the theoretical calculation and the actual machining of micro
  groove at both starting and ending points were compared. The influence of pulse voltage,
  mask size and scanning speed on the simulation results of micro groove was studied.

  (3) On the basis of the simulation results, the experimental verification was carried out.
  The relationship between the key parameters, such as pulse frequency, pulse duty cycle,
  scanning speed and scanning times, and the machining localization, material removal rate
  and current efficiency of the micro grooves was studied. Finally with the optimized
  parameters (pulse voltage 30 V, pulse frequency 1 000 Hz, pulse duty cycle 20\%, scanning
  speed 20 μm/s), serpentine, straight line, cross shape micro grooves and array microslit
  structure with the width of 90 μm, the depth of 20 μm were successfully generated, verified
  the feasibility of the masked jet electrochemical machining method.

  (4) In order to improve the localization of processed micro groove, the conductive mask
  was used to replace the insulation mask for the simulation and experimental study. The
  simulation results showed that using conductive mask could change the electric field
  distribution in the micro groove processing area, reduce the electric field identity at the edge
  of the mask hole, and reduce the undercutting of the profile compared to that generated with
  insulated mask. In the experiment, the etch factor ($F_{\symup{E}}$) increased from 1.47 to 6.11 using
  conductive mask, which showed a low undercutting and high machining localization.
\end{abstract*}

\gduttableofcontents

\mainmatter

\chapter{绪论}{Introduction}

\section{本课题研究背景及研究意义}{Background and significance of research}

随着科学技术的进步,产品逐渐向精密化和高性能化发展,具有毫米及微米尺度
微沟槽结构的金属零部件在国防军事、航空航天、新能源、新材料、生物医学、半导
体器件等领域的高技术产品中扮演的角色愈加重要。

\section{微沟槽电解加工国内外相关研究现状}{Analysis of the research status at home and abroad}

\subsection{成型电极电解加工}{Shaped cathode electrochemical machining}

采用与微沟槽结构形状对应的成型阴极,例如薄板阴极,片状阴极等,进行微沟
槽电解加工,其特点是方便一次成型微沟槽形状。南京航空航天大学吕焱明等进行了
大长宽比深窄槽电解加工阴极设计以及工艺试验研究$^{\text{[16]}}$,其采用的镂空片状阴极结构
内部带有加强筋增加刚度,外部非加工区需要绝缘处理。加工试验中片状阴极沿 Z 轴
向下进给,电解液从内部正向流入,对 TB6 钛合金加工得到长宽比 11:1、深宽比 9:1、
槽宽小于 3 mm 的直线、曲线深窄槽结构,如图 1-4 所示。

\subsection{电解铣削加工}{Electrochemical milling}

电解铣削是参照成熟的数控铣削加工,通常采用简单形状的微细工具电极,通过
控制其运动轨迹,能够实现复杂微结构的电解加工。同时配合超短脉宽脉冲电流或者
阴极辅助振动进给等技术后,可以很大程度上提高电解加工的溶解定域性$^{\text{[22]}}$。

\subsection{掩模电解加工}{Through-mask electrochemical machining}

掩模电解加工(through mask electrochemical micromachining,TMEMM),其原理
是在工件的表面涂敷一层光刻胶,经过光刻显影后,工件上形成具有一定图案的裸露
表面,然后通过电化学反应选择性地溶解未被光刻胶保护的裸露部分,最终加工出所
需形状$^{\text{[36]}}$,如图 1-14 所示。掩模电解加工具有蚀除速度快,加工效率高,加工质量好,
加工图案具有多样性的优点,是一种广泛使用的电解加工方式。

\subsection{射流电解加工}{Jet electrochemical machining}

\section{本课题研究目标和主要研究内容}{The research objectives and main contents of this subject}

\subsection{课题研究目标}{Research objectives}

电解加工相对于其它加工技术具有无工具损耗,无机械切削力,加工无毛刺等优
势,但传统电解加工方式还存在一些关键技术问题亟待解决,如掩模电解加工中掩模
制作工艺复杂、电解产物排除困难、沟槽长度方向尺寸一致性差;射流电解加工受到
喷嘴孔径限制加工尺度受到制约,容易产生杂散腐蚀等问题。针对上述问题本文提出
了一种模板射流电解加工微沟槽的新方法。该方法将带有单个或单排微通孔结构的模
板与金属喷嘴集成为一体,加工过程中保持模板与工件表面接触,电解液通过喷嘴高
速喷入模板微通孔到达工件表面,通过控制喷嘴与工件之间的相对运动便可由点到线
扫描加工出复杂曲线形或阵列微沟槽结构。相比传统电解加工方式,模板射流电解加
工方法具有加工定域性高,尺寸一致性好,加工重复性高并且加工成本低的优势。

针对该方法开展仿真模拟与实验研究,最终实现平面直线型,交叉型,弯曲蛇型
和螺旋形不同形状微沟槽结构高效、高精度、低成本电解加工,以满足微反应器、燃
料电池双极板以及微光栅等加工需求。

\subsection{主要研究内容}{Research content}

\section{课题来源}{Project source}

本课题来源于国家自然科学基金(U1601201,51705089)以及广州市珠江科技新
星专题项目(201906010099),并受以上基金资助。

\section{本章小结}{Chapter summary}

本章首先介绍了微沟槽结构的应用领域和场景,以及微沟槽结构加工制造的技术
难点,然后对电解加工微沟槽国内外相关研究现状进行综述,分析了现有电解加工方
法仍存在的不足之处,提出本文模板射流电解加工微沟槽课题研究目标和相关研究内
容。

\chapter{模板射流电解加工实验平台搭建}{Experimental platform of masked jet electrochemical machining}

本章对模板射流微沟槽实验加工系统进行介绍,主要包括平台运动控制和数据采
集系统软硬件搭建以及实验一体化模板喷嘴和工装夹具设计制作。实验平台的搭建是
进行后续实验研究的必要条件。

\section{加工原理与实验平台}{Processing principle and experimental platform}

\subsection{加工原理}{Processing principle}

采用贴有绝缘模板的金属喷嘴在工件表面射流电解加工微沟槽结构,加工原理如
\autoref{fig:example} 所示。将带有单个或单排微通孔结构的模板与金属喷嘴集成为一体,加工过程
中保持模板与工件表面接触,在喷嘴和工件间施加脉冲电压,电解液通过喷嘴高速喷
入模板微通孔到达工件表面,进行模板微通孔约束下的电解腐蚀,同时通过控制喷嘴
与工件之间的相对运动便可由点到线扫描加工出微沟槽结构。加工过程中将模板和工
件紧密贴合,削弱沿模板孔宽度方向腐蚀从而达到电解加工的定域蚀除效果。改变模
板孔径大小和数量即可加工出不同宽度的微沟槽阵列,通过编程控制模板喷嘴和工件
的相对运动轨迹即可加工出不同形状微沟槽。

\begin{figure}[h]
  \subfloat[贴有模板的金属喷嘴示意图]{\label{sub-fig-1}
    \includegraphics[width=0.4\textwidth]{example-image.pdf}
  }
  \qquad
  \subfloat[由点到线扫描加工原理图]{\label{sub-fig-2}
    \includegraphics[width=0.4\textwidth]{example-image.pdf}
  }
  \bicaption{模板射流电解加工微沟槽原理图}{Principle of masked jet electrochemical machining of micro grooves}
  \label{fig:example}
\end{figure}

整个加工过程如图 2-2 所示。加工开始时,如图 2-2(a)中带有绝缘模板的金属
喷嘴沿 Z 轴竖直向下运动贴近工件。图 2-2(b)模板喷嘴和工件贴合后电解液射流在
模板约束下流向工件表面,工件接电源正极,金属喷嘴接电源负极,在电源输出电压
下产生深度方向的电解腐蚀,同时模板喷嘴和工件相对运动,由点到线扫描加工出微
沟槽,如图 2-2(c)所示。加工完毕后关闭电源,停止通液,模板喷嘴沿 Z 轴抬起离
开工件表面,完成微沟槽加工过程,如图 2-2(d)所示。

\subsection{实验平台}{Experimental platform}

\section{实验平台运动控制系统}{Development of motion control system for experimental platform}

\subsection{运动控制系统功能要求}{Functional requirements of motion control system}

\subsection{运动控制系统整体架构}{Overall architecture of motion control system}

\subsection{基于 DMC 5400A 运动控制卡硬件组成}{Hardware based on DMC 5400A motion control card}

\begin{table}[h]
  \bicaption{DMC5400A 运动控制卡主要技术指标}{DMC5400A main specifications}
  \label{tab:example}
  \begin{tabular}{cc}
    \toprule
    控制卡技术指标              & 具体参数                      \\
    \midrule
    控制电机的脉冲信号频率范围  & $\SI{1}{Hz}\sim\SI{2}{MHz}$   \\
    控制电机的脉冲信号频率精度  & \SI{0.0625}{Hz}               \\
    脉冲信号输出最大电流        & \SI{20}{mA}                   \\
    脉冲信号长度                & 28 位有符号                   \\
    直线插补精度                & $\pm \SI{0.8}{pulse}$         \\
    圆弧插补精度                & $\pm \SI{1.5}{pulse}$         \\
    支持的插补坐标系个数        & 2                             \\
    \bottomrule
  \end{tabular}
\end{table}

\section{本章小结}{Chapter summary}

本章介绍了模板射流电解加工微沟槽的加工原理及实验平台,开发了基于
LabVIEW 软件的运动控制系统和数据采集系统,完成了模板喷嘴与工装夹具设计制作,
为后续进行模板射流电解加工微沟槽实验打下了基础。

\chapter{模板射流电解加工微沟槽成形规律}{Simulation analysis of microgroove processed by masked jet ECM}

本章利用 ANSYS 有限元仿真平台进行模板射流电解加工微沟槽成形动态仿真,模
拟微沟槽成形过程,从理论层面研究微沟槽成形规律,并通过仿真分析各参数对微沟
槽加工尺寸的影响,为后续实验验证理论分析结果打下基础。

\section{微沟槽加工过程建模}{Modeling of microgroove machining process}

采用模板射流电解加工方法加工出的微沟槽结构示意图如图 3-1 所示。由于微沟
槽为对称结构,为简化计算过程,将三维模型简化为二维模型,分别从微沟槽横截面
YOZ 与纵截面 XOY 两个平面进行单点微坑成形过程和微沟槽成形过程二维仿真计算。
单点微坑仿真中电解加工计算时间 $t$ 和微沟槽仿真中模板喷嘴扫描速度 $v$ 之间满足
\begin{align}
  t = d/v
\end{align}
其中 $d$ 为模板孔径。

根据欧姆定律,加工过程中工件表面电流密度 $i$ 与电场强度 $E$ 之间的关系为
\begin{align}
  i = \sigma E
\end{align}
其中 $σ$ 为电解液电导率。

根据法拉第定律,加工过程中电解腐蚀速度 $v_e$ 可以表示为
\begin{align}
  𝑣_e = 𝜂𝜔𝑖
\end{align}
其中 $η$ 为电流效率,$ω$ 为材料的体积电化学当量。

借鉴文献[76],采用质量称重法测定了电解加工过程中电流效率 $𝜂$ 和电流密度 $𝑖$ 之间
的关系为
\begin{equation}
  \eta = \frac{0.85}{1 + e^{(10-i)/6}} - 0.1
\end{equation}

因此加工过程中电解腐蚀速度 $v_e$ 与电场强度 $E$ 之间的关系可以表示为
\begin{align}
  𝑣_e = 𝜂𝜔𝜎𝐸
\end{align}

则电解加工深度 $h$ 与加工时间 $t$ 之间的关系为
\begin{align}
  ℎ = 𝑣_e 𝑡 = 𝜂𝜔𝜎𝐸𝑡
\end{align}

电解过程中加工区域的产物与焦耳热量在电解液的高速冲刷下可以被迅速带走,
因此仿真过程中假设有以下几个前提条件:

(1)电解液电导率 $σ$ 为常量;

(2)电解加工过程中温度 $T$ 恒定;

(3)加工过程中电解液的浓度不变。

\section{微沟槽加工成形过程动态仿真}{Dynamic simulation of microgroove forming process}

\subsection{单点微坑成形过程模拟}{Simulation of single point dimple forming process}

\subsection{微沟槽成形过程模拟}{Simulation of point-line microgroove forming process}

\subsection{微沟槽首尾两端形貌对比分析}{Analysis of forming transition arcs at both starting and ending points}

\section{微沟槽加工尺寸仿真结果分析}{Simulation results of microgroove processed by masked jet ECM}

\section{本章小结}{Chapter summary}

本章通过 ANSYS 电场仿真对模板射流电解加工微沟槽进行了理论分析。首先建立
了仿真理论模型,然后进行了单点微坑成形过程仿真计算,研究分析了加工区域电场
分布对微坑加工结果的影响,进而引入由点到线的微沟槽成形过程模拟,通过计算电
流密度变化过程分析了微沟槽加工首端圆角形成、扩展延伸进而形成微沟槽的三个阶
段,并对比分析了理论计算与实际加工微沟槽首尾两端形貌不同的原因。最后研究了
加工电压、模板尺寸与扫描速度不同参数对微沟槽仿真计算结果的影响,为进一步实
验与工艺参数优化打下了基础。

\chapter{模板射流电解加工微沟槽工艺实验}{Experimental analysis of microgroove processed by masked jet ECM}

本章在前文模板射流电解加工微沟槽成形规律研究的基础上进行微沟槽加工工艺
实验。针对微沟槽加工定域性、材料去除率以及电流效率等关键指标对脉冲频率、脉
冲占空比、扫描速度和扫描加工次数等工艺参数进行分析。最终采用优化后的参数组
合加工出弯曲蛇形、直线和交叉阵列形以及阵列微缝等各种复杂形状微沟槽结构。

\section{微沟槽加工关键评价指标}{Evaluation index of microgroove machining}

\subsection{加工定域性}{Processing localization}

\subsection{加工材料去除率}{Material removal rate}

\subsection{加工电流效率}{Current efficiency}

\section{微沟槽加工实验参数选定}{Selection of experimental parameters for microgroove machining}

\section{脉冲参数对加工结果的影响}{Pulse parameters analysis and optimization}

\section{扫描运动参数对加工结果的影响}{Scanning parameters analysis and optimization}

\begin{table}[h]
  \bicaption{不同扫描次数下微沟槽加工尺寸}{The size of micro groove machining under different scanning times}
  \begin{tabular}{*6{>{\centering\arraybackslash}p{2.2cm}}}
    \toprule
    扫描次数 $n$              & 扫描速度 $v$ /($μm·s^{-1}$) & 微沟槽宽度 $w$
    /μm & 微沟槽深度 $h$ /μm & 加工过切量 $u$ /μm & 侧向腐蚀系数 $F_{\symup{E}}$                   \\
    \midrule
    1 & 20 & 160.63 & 30.32 & 43.33 & 1.43 \\
    2 & 40 & 160.97 & 30.48 & 44.03 & 1.44 \\
    3 & 60 & 161.57 & 30.78 & 45.10 & 1.46 \\
    4 & 80 & 161.90 & 30.95 & 45.70 & 1.48 \\
    \bottomrule
  \end{tabular}
\end{table}


\section{典型微沟槽结构加工实验}{Experiment of typical microgroove machining}

\section{本章小结}{Chapter summary}

本章在前文仿真计算结果分析的基础上进行实验验证,采用模板射流电解加工微
沟槽方法,研究了影响加工结果的关键参数如脉冲频率、占空比以及喷嘴扫描速度和
扫描次数与微沟槽加工尺寸、加工定域性、材料去除率以及电流效率之间的关系,得
出的结论如下。

(1)脉冲占空比对加工微沟槽过切量和深度影响比较显著,而脉冲频率对加工尺
寸影响不明显。当减小脉冲占空比时加工微沟槽过切量和深度均减小,20\%占空比下
加工微沟槽过切量最小。当脉冲占空比从 20\%增大到 80\%,相应的有效加工时间增加,
但脉冲间隔时间减少,不利于加工过程中电解液更新和电解产物的排除,并且有效材
料去除率和电流效率降低。优化后选用的脉冲参数为脉冲频率 1 000 Hz,占空比 20\%。

(2)喷嘴扫描速度与微沟槽加工槽深槽宽以及过切量呈反比关系,即扫描速度越
快加工出的微沟槽宽度及过切量越小,但也会导致微沟槽加工深度过浅,相应的侧向
腐蚀系数降低,即加工定域性降低。当扫描速度增加的同时进行多次扫描,在保持有
效加工时间一致的条件下可以达到与较低速度扫描相同的加工结果。

(3)针对微反应器、微流控芯片、微型燃料电池双极板以及微光栅中微沟槽结构
加工制造难题,采用模板射流电解加工方法,通过改变模板孔径和模板间距,并优选
参数组合加工出弯曲蛇形、直线阵列形和交叉阵列形微沟槽,以及直线阵列微缝结构,
为各种复杂形状微沟槽结构高精度高效低成本加工提供了新方法。

\chapter{采用金属模板提高射流加工定域性的探索}{Using Conductive mask to improve the localization of masked jet ECM}

本章首先采用金属模板完成单点微坑电解加工仿真计算,从理论层面分析其高加
工定域性原因并进行实验验证,最后进行了金属模板射流电解加工微沟槽的实验探索。

\section{金属模板提高加工定域性理论分析}{Analysis of conductive mask improving localization of machining}

\subsection{金属模板加工区域电场分布规律}{Electric field distribution in conductive mask processing area}

\subsection{金属模板加工结果仿真模拟}{Simulation of conductive mask machining results}

\section{金属模板与绝缘模板加工成形结果对比}{Comparison of forming results using conductive and insulation mask}

\subsection{相同深度下加工结果对比}{Comparison of processing results at the same depth}

\subsection{不同深度下加工结果对比}{Comparison of processing results at different depths}

\section{金属模板螺旋形微沟槽结构加工实验}{Experiments on machining spiral microgroove of conductive mask}

\section{本章小结}{Chapter summary}

为提高微沟槽模板射流电解加工定域性,本章采用金属模板代替绝缘模板进行相
关仿真计算和实验研究。主要包括以下内容。

\gdutbackmatter

\chapter{结论与展望}{Conclusion and prospect}

\gdutbacksection{研究结论}

微沟槽作为一种典型微结构在微传感器、微芯片、微流体器件以及微型燃料电池
双极板等产品核心部件上发挥重要作用。在微沟槽加工方面电解加工具有无应力接触、
无电极损耗,可实现离子级别的材料去除的优势。针对现有掩模电解加工中掩模制作
工艺复杂、电解产物排除困难、沟槽长度方向尺寸一致性差等问题,本文提出一种模
板射流电解加工微沟槽结构新方法,并进行相关理论和实验研究。论文完成的主要工
作如下。

(1)提出模板射流电解加工方法,搭建电解加工微沟槽实验平台。完成模板射流
电解加工喷嘴与工装夹具设计制作,开发基于 LabVIEW 软件的运动控制系统和数据采
集系统,以满足实际加工实验需求。

(2)通过 ANSYS 电场仿真对模板射流电解加工微沟槽成形规律进行理论分析。
分别完成了单点微坑成形过程仿真计算和由点到线微沟槽成形过程模拟,通过计算电
流密度变化过程分析了微沟槽加工首尾两端圆角形成过程,仿真计算结果表明,微沟
槽在 15 s 时首端圆角形成并达到最大加工深度。将理论计算与实际加工微沟槽首尾两
端形貌进行对比,表明电解产物堆积是产生微沟槽首尾两端形貌不同的因素。研究了
加工电压、模板尺寸与扫描速度参数对微沟槽仿真计算结果的影响。

(3)进行了模板射流电解加工微沟槽工艺参数优化。研究了影响加工结果的关键
参数如脉冲频率、占空比以及喷嘴扫描速度和扫描次数与微沟槽加工尺寸一致性、加
工定域性、材料去除率以及电流效率之间的关系。研究结果表明脉冲占空比是影响加
工结果的重要因素而脉冲频率对加工结果影响不明显。最终通过参数优选,在脉冲电
源电压 30 V、脉冲频率 1 000 Hz,占空比 20\%,扫描速度 20 μm/s 的条件下成功加工
出最小槽宽 90 μm,最小槽深 20 μm 的弯曲蛇形、直线和交叉阵列形微沟槽结构,以
及阵列微缝结构,可应用于微反应器、微流控芯片、微型燃料电池双极板以及微光栅
等器件,验证了模板射流电解加工方法的可行性。

(4)采用金属模板进行了提高射流加工定域性的仿真计算和实验研究。仿真结果
表明采用金属模板能够改变微沟槽加工区域电场分布,使模板孔边缘处电场显著削弱,
从而减小侧向腐蚀。相同加工深度下绝缘模板和金属模板加工对比实验结果表明,相
比绝缘模板,采用金属模板加工侧向腐蚀系数 $F_{\symup{E}}$ 从 1.47 直接增加至 6.11。采用金属模
板能够减小射流加工微沟槽侧向腐蚀过切量,显著改善加工定域性。

\gdutbacksection{未来研究展望}
本文针对模板射流电解加工微沟槽进行了初步的理论分析和实验研究,由于本课
题涉及的范围广泛,加上时间和能力有限,未能进行更加详细和深入的研究。后续可
以从以下几个方面进行进一步研究。

(1)本文采用模板射流电解加工方法加工出平面微沟槽结构,通过改进实验加工
装置增加运动轴数,可以考虑进行数控曲面微沟槽加工。

(2)实际电解加工需要综合考虑电场、流场以及电化学溶解速度场等多场因素,
本文仅针对电场进行微沟槽成形过程仿真,后续可以进一步探索微沟槽多场耦合理论
仿真分析。

(3) 针对更广范围内的微沟槽结构比如刀具表面阵列沟槽形微织构进行加工实
验,探索阵列沟槽形表面微织构的摩擦磨损特性。

\chapter{攻读学位期间取得与学位论文相关的成果}{Publication and patents during study}

\gdutbacksection{发表和投稿与学位论文相关学术论文}

\begin{results}
  \item \textbf{张三}, 李四, 王五, 等. Jet electrochemical machining of micro dimples with conductive mask.
  Journal of Materials Processing Technology. 2018, 257:101-111. (SCI Impact Factor 3.647,
  WOS:000431161400010)
  \item 李四, \textbf{张三}, 王五, 等. Electrochemical direct-writing machining of micro- channel array.
  Journal of Materials Processing Technology. 2019, 265:138-149. (SCI Impact Factor 3.647,
  WOS:000451935100014)
\end{results}

\gdutbacksection{申请发明专利}

\begin{results}
  \item 李四, \textbf{张三}, 王五. 一种微流道电解加工装置. 发明专利申请号: 201810467763.5.
\end{results}

\gdutstatement

\chapter{致谢}{Acknowlegements}

行文至此,感慨万千。

\end{document}