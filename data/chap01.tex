\chapter{绪论}{Introduction}

\section{本课题研究背景及研究意义}{Background and significance of research}

\zhlipsum[1]

\section{国内外相关研究现状}{Analysis of the research status at home and abroad}

\subsection{测试}{test}

测试\gdutcite{chendengyuan2000guoshi},测试\gdutcite*{woerdelun2012jingji}。

测试如\autoref{fig:example} 所示,具体参考\autoref{sub-fig-1} 和\autoref{sub-fig-2}。

测试参考\autoref{eq:example}。

测试参考\autoref{tab:example}。

测试\gls{slm},\gls{slm},\gls{glm},\gls{glm}。

\subsubsection{测试}
测试 test。
\paragraph{测试 test。}
测试 test。
\subparagraph{测试 test。}
测试 test。

% 大部分情况使用 align/gather 环境,小部分间距异常情况可尝试使用 equation 环境
\begin{align}
  E &= mc^2 \label{eq:example} \\
  mc^2 &= E
\end{align}

\begin{gather}
  E = mc^2 \\
  mc^2 = E
\end{gather}

\begin{figure}
  \subfloat[贴有模板的金属喷嘴示意图]{\label{sub-fig-1}
    \includegraphics[width=0.4\textwidth]{example-image.pdf}
  }
  \qquad
  \subfloat[由点到线扫描加工原理图]{\label{sub-fig-2}
    \includegraphics[width=0.4\textwidth]{example-image.pdf}
  }
  \bicaption{模板射流电解加工微沟槽原理图}{Principle of masked jet electrochemical machining of micro grooves}
  \label{fig:example}
\end{figure}

\begin{table}
  \bicaption{DMC5400A 运动控制卡主要技术指标}{DMC5400A main specifications}
  \label{tab:example}
  \begin{tabular}{cc}
    \toprule
    控制卡技术指标              & 具体参数                      \\
    \midrule
    控制电机的脉冲信号频率范围  & $\SI{1}{Hz}\sim\SI{2}{MHz}$   \\
    控制电机的脉冲信号频率精度  & \SI{0.0625}{Hz}               \\
    脉冲信号输出最大电流        & \SI{20}{mA}                   \\
    脉冲信号长度                & 28 位有符号                   \\
    直线插补精度                & $\pm \SI{0.8}{pulse}$         \\
    圆弧插补精度                & $\pm \SI{1.5}{pulse}$         \\
    支持的插补坐标系个数        & 2                             \\
    \bottomrule
  \end{tabular}
\end{table}